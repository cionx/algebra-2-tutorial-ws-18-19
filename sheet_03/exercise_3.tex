\section{}





\subsection{}

The ideal~$(xy)$ is homogeneous because it is generated by the homogeneous element~$xy \in k[x,y]$.
Hence the quotient~$k[x,y]/(xy)$ inherits a grading from~$k[x,y]$ via the canonical projection~$\pi \colon k[x,y] \to k[x,y]/(xy)$.
To be more precise, we have for~$A \defined k[x,y]/(xy)$ that
\[
    A_d
  = \pi(k[x,y]_d)
\]
for every~$d \geq 0$.
The ideal
\[
    (xy)
  = \{
      f \cdot xy
    \suchthat
      f \in k[x,y]
    \}
\]
has as a basis the monomials~$x^n y^m$ with~$n, m \geq 1$, hence the quotient~$A = k[x,y]/(xy)$ has as a basis the residue classes
\[
  1, \class{x}^n, \class{y}^m
  \qquad
  \text{with~$n, m \geq 1$} \,.
\]
It follows that~$A_0$ is \dash{one}{dimensional} (with single basis element~$1$) whereas~$A_d$ is \dash{two}{dimensional} (with basis elements~$\class{x}^d, \class{y}^d$) for every~$d \geq 0$.
The Hilbert~series of~$A$ is hence given by
\[
    1 + 2t + 2t^2 + 2t^3 + \dotsb
  = 1 + 2 \sum_{n=1}^\infty t^n
  = 1 + 2t \sum_{n=0}^\infty t^n
  = 1 + \frac{2t}{1-t}
  = \frac{1+t}{1-t} \,.
\]





\subsection{}

We have for all~$d \geq 0$ that
\[
    \dim(\Tensor(V)_d)
  = \dim(V^{\tensor d})
  = \dim(V)^d
  = n^d \,.
\]
The Hilbert~series of~$\Tensor(V)$ is hence given by
\[
    \sum_{n=0}^\infty n^d t^d
  = \sum_{n=0}^\infty (nt)^d
  = \frac{1}{1-nt} \,.
\]





\subsection{}

We first note that the {\twosided} ideal
\[
            I
  \defined  \genideal{
              x \tensor y - y \tensor x
            \suchthat
              x, y \in V
            }
\]
is generated by homogeneous elements and is therefore homogeneous.
The quotient algebra~$\Tensor(V)/I$ therefore inherits a grading from~$\Tensor(V)$, which is given by
\[
    ( \Tensor(V)/I )_d
  = \pi( \Tensor(V)_d )
  = \pi( V^{\tensor d} )
\]
for all~$d \geq 0$, where~$\pi \colon \Tensor(V) \to \Tensor(V)/I$ denotes the canonical projection.

To further determine the quotient algebra~$\Tensor(V)/I$ we proceed similarly to Exercise~3, part~(b) of the second exercise sheet by showing that~$\Tensor(V)/I \cong \Symm(V)$:

It follows from the universal property of the tensor algebra~$\Tensor(V)$ that the inclusion~$V \to \Symm(V)$ induces an algebra homomorphism
\[
          \tilde{\varphi}
  \colon  \Tensor(V)
  \to     \Symm(V) \,,
\]
which is given on simple tensors by
\[
    \tilde{\varphi}(v_1 \tensor \dotsb \tensor v_n)
  = v_1 \dotsm v_n \,.
\]
It hold that~$I \subseteq \ker(\tilde{\varphi})$ because the algebra~$\Symm(V)$ is commutative.
The algebra homomorphism~$\tilde{\varphi}$ therefore induces a {\welldef} algebra homomorphism
\[
          \varphi
  \colon  \Tensor(V)/I
  \to     \Symm(V)
\]
which is on residue classes of simple tensors given by
\[
    \varphi( \class{v_1 \tensor \dotsb \tensor v_n} )
  = v_1 \dotsm v_n \,.
\]

To construct an inverse to~$\varphi$ we note that the algebra~$\Tensor(V)/I$ is generated by the residue classes~$\class{v}$ with~$v \in V$ (because the tensor algebra~$\Tensor(V)$ is generated by the elements~$v \in V$) and that these generators commute in the quotient~$\Tensor(V)/I$ because
\[
    \class{v_1} \, \class{v_2} - \class{v_2} \, \class{v_1}
  = \class{v_1 \tensor v_2 - v_2 \tensor v_1}
  = 0
\]
for all~$v_1, v_2 \in V$.
The quotient algebra~$\Tensor(V)$ is therefore commutative.
It follows from the universal property of the symmetric algebra~$\Symm(V)$ that the linear map
\[
          V
  \to     \Tensor(V)
  \to     \Tensor(V)/I \,,
  \quad   v
  \mapsto \class{v}
\]
induces a {\welldef} algebra homomorphism
\[
          \psi
  \colon  \Symm(V)
  \to     \Tensor(V) \,,
\]
which is given on monomials by
\[
    \psi( v_1 \dotsm v_n )
  = \class{v_1} \dotsm \class{v_n}
  = \class{v_1 \tensor \dotsb \tensor v_n}
\]
for all~$v_1, \dotsc, v_n \in V$.

The two algebra homomorphisms~$\varphi$ and~$\psi$ are mutually inverse on the algebra generators~$\class{v}$ of~$\Tensor(V)$ and~$v$ of~$\Symm(V)$, where~$v \in V$, and are hence mutually inverse.
In other words, the algebra homomorphism~$\varphi$ is an isomorphism with~$\varphi^{-1} = \psi$.

The algebra homomorphism~$\tilde{\varphi} \colon \Tensor(V) \to \Symm(V)$ maps for all~$d \geq 0$ the homogeneous component~$\Tensor(V)_d = V^{\tensor d}$ onto the homogeneous component~$\Symm^d(V)$.
It follows that the induced homomorphism~$\varphi \colon \Tensor(V)/I \to \Symm(V)$ also maps the homogeneous component~$(\Tensor(V)/I)_d$ onto the homogeneous component~$\Symm^d(V)$.
The algebra isomorphism~$\varphi$ is therefore aready an isomorphism of graded~{\kalgs}.

\begin{remark}
  Let~$J = \genideal{xy - yx \suchthat x, y \in \Tensor(V)}$ be the commutator ideal of~$\Tensor(V)$ from Exercise~3, part~(b) of the second exercise sheet.
  It holds that~$I \subseteq J$, and because the quotient~$\Tensor(V)/I$ is commutative it also holds that~$J \subseteq I$.
  (Recall that the commutator ideal~$I$ is the smallest {\twosided} ideal in~$A$ whose quotient is commutative.)
  Together this shows that~$I = J$.
  We could have therefore also referred to Exercise~3, part~(b) of the second exercise sheet to conclude that~$\Tensor(V) \cong \Symm(V)$ as graded~{\kalgs}.
\end{remark}

By choosing a basis~$v_1, \dotsc, v_n$ of the vector space~$V$ we can further identify the symmetric algebra~$\Symm(V)$ with the polynomial ring~$k[x_1, \dotsc, x_n]$ as graded~{\kalgs}.
(See the solutions to Exercise~3, part~(b) of the second exercise sheet for the detailed calculations.)
Altogether we have that
\[
        \Tensor(V)/I
  \cong \Symm(V)
  \cong k[x_1, \dotsc, x_n]
\]
as graded~{\kalgs}.
The Hilbert series of~$\Tensor(V)/I$ is therefore the same as the one of~$k[x_1, \dotsc, x_n]$, which we can compute and express in two possible ways:

\begin{itemize}
  \item
    By using the counting method \enquote{stars and bars} we can compute for every~$d \geq 0$ that
    \[
        \dim k[x_1, \dotsc, x_n]_d
      = \binom{d + n-1}{n-1} \,.
    \]
    The searched Hilbert series is therefore given by
    \[
      \sum_{d=0}^\infty \binom{d + n-1}{n-1} t^n \,.
    \]
  \item
    We may use that
    \[
            k[x_1, \dotsc, x_n]
      \cong k[x_1] \tensor \dotsb \tensor k[x_n]
    \]
    as graded~{\kalgs}, and that the Hibert series of~$k[x]$ is given by
    \[
        1 + t + t^2 + t^3 + \dotsb
      = \frac{1}{1-t} \,.
    \]
    It then follows that the Hilbert series of~$k[x_1, \dotsc, x_n]$ is given by
    \[
        \underbrace{\frac{1}{1-t} \, \dotsm \, \frac{1}{1-t}}_{\text{$n$ times}}
      = \frac{1}{(1-t)^n} \,.
    \]
\end{itemize}






