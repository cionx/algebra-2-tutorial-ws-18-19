\section{}





\subsection{}
\label{grading on tensor algebra}

We already know that~$\Tensor(V)$ is a~{\kalg} with a decomposition~$\Tensor(V) = \bigoplus_{d \geq 0} V^{\tensor d}$ into linear subspaces.
It remains to show that this decomposition gives a grading on~$\Tensor(V)$, i.e.\ that for holds for all~$x \in V^{\tensor d}$ and~$x' \in V^{\tensor d'}$ that~$x x' \in V^{\tensor (d + d')}$.
For this we may assume that~$x$ and~$x'$ are simple tensors
\[
    x
  = v_1 \tensor \dotsb \tensor v_d
  \quad\text{and}\quad
    x'
  = v'_1 \tensor \dotsb \tensor v'_{d'} \,.
\]
Then
\begin{align*}
      x x'
  &=  (v_1 \tensor \dotsb \tensor v_d) (v'_1 \tensor \dotsb \tensor v'_{d'})  \\
  &=  v_1 \tensor \dotsb \tensor v_d \tensor v'_1 \tensor \dotsb \tensor v'_{d'}
  \in V^{\tensor (d + d')} \,.
\end{align*}


\begin{remark}
  We find in the same same way as for the tensor algebra~$\Tensor(V)$ that both the symmetric algebra~$\Symm(V) = \bigoplus_{d \geq 0} \Symm^d(V)$ and the exterior algebra~$\Exterior(V) = \bigoplus_{d \geq 0} \Exterior^d(V)$ are graded~{\kalgs}.
\end{remark}





\subsection{}

% We recall the following universal properties from (linear) algebra:
% 
% \begin{itemize}
%   \item
%     The universal property of the tensor algebra~$\Tensor(V)$:
%     
%     If~$V$ is a~{\kvs} then the inclusion
%     \[
%                   \iota
%       \colon      V
%       =           V^{\tensor 1}
%       \inclusion  \Tensor(V) \,,
%     \]
%     is a~{\klin} map, and for every other~{\kalg}~$A$ and every~{\klin} map~$f \colon V \to A$ there exists a unique homomorphism of~{\kalg}~$F \colon \Tensor(V) \to A$ which extends~$f$ along~$\iota$, i.e.\ which makes the triangle
%     \[
%       \begin{tikzcd}[sep = large]
%           \Tensor(V)
%           \arrow[dashed]{r}[above]{F}
%         & A
%         \\
%           V
%           \arrow{u}[left]{\iota}
%           \arrow{ur}[below right]{f}
%         & {}
%       \end{tikzcd}
%     \]
%     commute.
%     The algebra homorphism~$F$ is on simple tensors given by
%     \[
%         F(v_1 \tensor \dotsb \tensor v_d)
%       = f(v_1) \dotsm f(v_d)
%     \]
%     for all~$v_1, \dotsc, v_d \in V$.
%     This constructions results in a {\onetoone} correspondence
%     \[
%         \{ \text{algebra homomorphisms~$F \colon \Tensor(V) \to A$} \}
%       \longleftrightarrow
%         \{ \text{linear maps~$f \colon V \to A$} \} \,.
%     \]
%     (This means in fancy language that~$\Tensor \colon k\text{-}\mathbf{Vect} \to k\text{-}\mathbf{Alg}$ is left adjoint to the forgetful functor~$k\text{-}\mathbf{Alg} \to k\text{-}\mathbf{Vect}$.
%     The inclusion~$\iota \colon V \to \Tensor(V)$ is the unit of this adjunction.)
%   \item
%     The universal property of the symmetric algebra~$\Symm(V)$:
%     
%     If~$V$ is a~{\kvs}, then the inclusion
%     \[
%                   \iota
%       \colon      V
%       =           \Symm^1(V)
%       \inclusion  \Symm(V)
%     \]
%     is a~{\klin} map, and for every other commutative~{\kalg}~$C$ and every~{\klin} map~$f \colon V \to C$ there exists a unique homomorphism of~{\kalgs}~$F \colon \Symm(V) \to C$ which extends~$f$ along~$\iota$, i.e.\ which makes the triangle
%     \[
%       \begin{tikzcd}[sep = large]
%           T(V)
%           \arrow{r}[above]{F}
%         & C
%         \\
%           V
%           \arrow{u}[left]{\iota}
%           \arrow{ur}[below right]{f}
%         & {}
%       \end{tikzcd}
%     \]
%     commute.
%     The algebra homomorphism~$F$ is on monomials given by
%     \[
%         F(v_1 \dotsm v_d)
%       = f(v_1) \dotsm f(v_d)
%     \]
%     for all~$v_1, \dotsc, v_d \in V$.
%     This constructions results in a {\onetoone} correspondence
%     \[
%         \{ \text{algebra homomorphisms~$F \colon \Symm(V) \to C$} \}
%       \longleftrightarrow
%         \{ \text{linear maps~$f \colon V \to C$} \} \,.
%     \]
%     (This means in fancy language that~$\Symm \colon k\text{-}\mathbf{Vect} \to k\text{-}\mathbf{CommAlg}$ is left adjoint to the forgetful functor~$k\text{-}\mathbf{CommAlg} \to k\text{-}\mathbf{Vect}$.
%     The inclusion~$\iota \colon V \to \Symm(V)$ is the unit of this adjunction.)
%   \item
%     The universal property of the quotient ring~$R/I$:
%     
%     If~$R$ is a ring and~$I \subseteq R$ is a {\twosided} ideal, the the canonical projection
%     \[
%               \pi
%       \colon  R
%       \to     R/I \,,
%       \quad   r
%       \mapsto [r]
%     \]
%     is a ring homomorphism with~$\ker \pi = I$.
%     If~$S$ is any other ring and~$f \colon R \to S$ is a ring homomorphism with~$I \subseteq \ker(f)$ then~$f$ factors uniquely through a ring homomorphism~$\bar{f} \colon R/I \to S$, i.e\ the ring homomorphism~$\bar{f}$ makes the triangle
%     \[
%       \begin{tikzcd}[sep = large]
%           R
%           \arrow{r}[above]{f}
%           \arrow{d}[left]{\pi}
%         & S
%         \\
%           R/I
%           \arrow[dashed]{ur}[below right]{\bar{f}}
%         & {}
%       \end{tikzcd}
%     \]
%     commute.
%     The ring homomorphism~$\bar{f}$ is given by
%     \[
%         \bar{f}([x])
%       = f(x)
%     \]
%     for every~$[x] \in R/I$.
%     This construction results in a {\onetoone} correspondence
%     \begin{align*}
%         &\{ \text{ring homomorphisms~$\bar{f} \colon R/I \to S$} \}
%       \\
%       \longleftrightarrow{}
%         &\{ \text{ring homomorphisms~$f \colon R \to S$ with~$I \subseteq \ker(f)$} \} \,.
%     \end{align*}
%   \item
%     The universal propery of the (commutative) polynomial ring~$k[X_i \suchthat i \in I]$:
%     
%     If~$I$ is any (index) set and~$C$ is a commutative (!)~{\kalg}, then there exist for any choice of elements~$c_i \in C$ with~$i \in I$ a unique algebra homomorphism~$F \colon k[X_i \suchthat i \in I] \to C$ with~$f(X_i) = c_i$ for every~$i \in I$.
%     
%     This means that for the \enquote{inclusion}
%     \[
%               \iota
%       \colon  I
%       \to     k[X_i \suchthat i \in I] \,,
%       \quad   i
%       \mapsto X_i
%     \]
%     every map~$f \colon I \to C$ extends uniquely to a algebra homomorphism~$F \colon k[X_i \suchthat i \in I]$ along~$\iota$, i.e.\ the homomorphism~$F$ makes the triangle
%     \[
%       \begin{tikzcd}[sep = large]
%           k[X_i \suchthat i \in I]
%           \arrow[dashed]{r}[above]{F}
%         & C
%         \\
%           I
%           \arrow{u}[left]{\iota}
%           \arrow{ur}[below right]{f}
%         & {}
%       \end{tikzcd}
%     \]
%     commute.
%     The algebra homomorphism~$F$ is given on monomials by
%     \[
%         F(X_{i_1}^{n_1} \dotsm X_{i_r}^{n_r})
%       = f(x_{i_1})^{n_1} \dotsm f(x_{i_r})^{n_r}
%       = c_{i_1}^{n_1} \dotsm c_{i_r}^{n_r} \,.
%     \]
%     This construction results in a {\onetoone} correspondence
%     \begin{align*}
%         &\{ \text{algebra homomorphisms~$F \colon k[X_i \suchthat i \in I] \to C$} \} \\
%       \longleftrightarrow{}
%         &\{ \text{tupels~$(c_i)_{i \in I}$ with~$c_i \in C$} \} \,.
%     \end{align*}
%     (This means in fancy language that the polynomial ring (as a functor) is a left adjoint to the forgetful functor~$k\text{-}\mathbf{CommAlg} \to \mathbf{Set}$.
%     The map~$i \mapsto X_i$ is the unit of this adjunction.)
% \end{itemize}

Let~$A$ be a~{\kalg}.
Then for every two elements~$x, y \in A$ we denote by
\[
            [x,y]
  \defined  xy - yx
\]
their \emph{commutator}.
As a map~$[-,-] \colon A \times A \to A$ the commutator is both {\kbil} and alternating.%
\footnote{It is in fact a Lie bracket, and one of the prototypical examples of such.}
We can now consider the {\twosided} ideal~$I$ in~$A$ given by
\[
    I
  = \genideal{ [x,y] \suchthat x, y \in A } \,.
\]
The ideal~$I$ is known as the \emph{commutator ideal} of~$A$.

Suppose now that~$A = \bigoplus_{d \geq 0} A_d$ is a graded~{\kalg}.
If~$x \in A_d$ and~$y \in A_{d'}$ are homogenous elements of~$A$ then their commutator~$[x,y]$ is again homogeneous because
\[
      [x,y]
  =   xy - yx
  \in A_{d + d'} + A_{d + d'}
  =   A_{d + d'} \,.
\]
It follows that the commutator ideal~$I$ of~$A$ is a homogeneous ideal:
We have for any two elements~$x, y \in A$ with homogeneous decompositions~$x = \sum_{d \geq 0} x_d$ and~$y = \sum_{d \geq 0} y_d$ that
\[
    [x,y]
  = \left[
      \sum_{d \geq 0} x_d,
      \sum_{d \geq 0} y_d
    \right]
  = \sum_{d, d' \geq 0} [x_d, y_{d'}]
\]
by the bilinearity of the commutator~$[-,-]$.
We therefore find that the commutator ideal
\[
    I
  = \genideal{
      [x,y]
    \suchthat 
      x, y \in A
    }
  = \genideal{
      [x', y']
    \suchthat
      d, d' \geq 0,
      x' \in A_d,
      y' \in A_{d'}
    }
\]
is generated by homogeneous elements, and is hence homogeneous.
It follows that the quotient algebra~$A/I$ inherits a grading from~$A$;
to be more precise, we have that~$A/I = \bigoplus_{d \geq 0} \pi(A_d)$, where~$\pi \colon A \to A/I$ denotes the canonical projection.

\begin{remark}
  Let~$A$ be a~{\kalg} and let~$I$ be the commutator ideal of~$A$.
  Then the quotient algebra~$A/I$ is commutative, and~$I$ is the smallest ideal in~$A$ with this property.
  Indeed, it holds for any {\twosided} ideal~$J \subseteq A$ that
  \begin{align*}
        {}& \text{$A/J$ is commutative} \\
    \iff{}& \text{$x' y' = y' x'$ for all~$x', y' \in A/J$} \\
    \iff{}& \text{$\class{x} \, \class{y} = \class{y} \, \class{x}$ for all~$x, y \in A$} \\
    \iff{}& \text{$\class{xy - yx} = 0$ for all~$x, y \in A$} \\
    \iff{}& \text{$[x,y] \in J$ for all~$x, y \in A$} \\
    \iff{}& I \subseteq J \,.
  \end{align*}
  It follows (both from this result itself, and from a similiar calculation) that if~$C$ is a commutative~{\kalg} then there exists for every algebra homomorphism~$f \colon A \to C$ a unique algebra homomorphism~$\induced{f} \colon A/I \to C$ which makes the triangle
  \[
    \begin{tikzcd}
        A/I
        \arrow[dashed]{r}[above]{\induced{f}}
      & C
      \\
        A
        \arrow{u}
        \arrow{ur}[below right]{f}
      & {}
    \end{tikzcd}
  \]
  commute.
  This gives rise to a {\onetoone} correspondence
  \begin{align*}
      {}&\{ \text{algebra homomorphisms~$f \colon A \to C$} \}  \\
    \longleftrightarrow{}&
      \{ \text{algebra homomorphisms~$\induced{f} \colon A/I \to C$} \} \,.
  \end{align*}
  (We have thus constructed a left adjoint to the forgetful functor~$\kCommAlg \to \kAlg$.
  The author encourages the reader to compare this construction to the abelianization~$G/[G,G]$ of a group~$G$.)
  Hence the quotient algebra~$A/I$ is the \enquote{most general way} to make the algebra~$A$ commutative.
\end{remark}

We now return to the original algebra~$\Tensor(V)$ and its commutator ideal~$I$.
Thinking about the algebra~$\Tensor(V)$ as the \enquote{most general way of making the vector space~$V$ into a {\kalg}} and about the quotient~$\Tensor(V)/I$ as the \enquote{most general way to make the algebra~$\Tensor(V)$ commutative}, we may guess that the quotient algebra~$\Tensor(V)/I$ should be the \enquote{most general way to make the vector space~$V$ into a commutative~{\kalg}}.
In this way we see that the quotient algebra~$\Tensor(V)/I$ ought to be the symmetric algebra~$\Symm(V)$.

Indeed, the inclusion~$i \colon V \to \Symm(V)$ is a~{\klin} map and hence induces by the universal property of the tensor algebra~$\Tensor(V)$ an algebra homomorphism
\[
          \tilde{\varphi}
  \colon  \Tensor(V)
  \to     \Symm(V) \,,
\]
which is given on simple tensors by
\[
    \tilde{\varphi}(v_1 \tensor \dotsb \tensor v_n)
  = v_1 \dotsm v_n \,.
\]
It follows from~$\Symm(V)$ being commutative that~$\varphi$ extends to an algebra homomorphism
\[
          \varphi
  \colon  \Tensor(V)/I
  \to     \Symm(V) \,,
\]
which is given by
\[
    \varphi( \class{v_1 \tensor \dotsb \tensor v_n} )
  = v_1 \dotsm v_n
\]
for all~$v_1, \dotsc, v_n$.
It follows on the other hand from~$\Tensor(V)/I$ being commutative that the {\klin} map
\[
          j
  \colon  V
  \to     \Tensor(V)
  \to     \Tensor(V)/I \,,
  \quad   v
  \mapsto \class{v}
\]
induces an algebra homomorphism
\[
          \psi
  \colon  \Symm(V)
  \to     \Tensor(V)/I \,,
\]
which is on monomials given by
\[
    \psi(v_1 \dotsm v_n)
  = \class{v_1} \dotsm \class{v_n}
  = \class{v_1 \tensor \dotsb \tensor v_n}
\]
for all~$v_1, \dotsc, v_n \in V$.
The algebra homomorphisms~$\varphi$ and~$\psi$ are mutually inverse on vector space generators (the residue classes~$\class{v_1 \tensor \dotsb \tensor v_n}$ for~$\Tensor(V)/I$ and the monomials~$v_1 \dotsm v_n$ for~$\Symm(V)$), and are hence mutually inverse.
In other words,~$\varphi$ is an isomorphism with~$\varphi^{-1} = \psi$.


\begin{remark}
  One could also reformulate the above discussion by noting that for every commutative~{\kalg}~$C$ we have bijections
  \begin{align*}
         {}&  \{ \text{algebra homomorphisms~$\Tensor(V)/I \to C$} \} \\
    \cong{}&  \{ \text{algebra homomorphisms~$\Tensor(V) \to C$} \} \\
    \cong{}&  \{ \text{linear maps~$V \to C$} \}  \\
    \cong{}&  \{ \text{algebra homomorphisms~$\Symm(V) \to C$} \}
  \end{align*}
  by the universal property of the quotient algebra~$\Tensor(V)/I$, the tensor algebra~$\Tensor(V)$, and the symmetric algebra~$\Symm(V)$.
  These bijections are actually natural in~$C$, and hence give a natural isomorphism between convariant~\dash{$\Hom$}{functors}
  \[
          \Hom_{\kCommAlg}(\Tensor(V)/I, -)
    \cong \Hom_{\kCommAlg}(\Symm(V), -) \,.
  \]
  It now follows from Yoneda’s lemma that~$\Tensor(V)/I \cong \Symm(V)$.
\end{remark}

Note that the constructed homomorphism~$\tilde{\varphi} \colon \Tensor(V) \to \Symm(V)$ maps the homogeneous component~$\Tensor(V)_d = V^{\tensor d}$ onto the homogeneous component~$\Symm^d(V)$, and is therefore a homomorphism of graded~{\kalgs}.
It follows that the induced isomorphism~$\varphi \colon \Tensor(V)/I \to \Symm(V)$ is also a homomorphism of~{\kalgs} because
\[
    \varphi( (\Tensor(V)/I)_d )
  = \varphi( \pi(\Tensor(V)_d) )
  = \tilde{\varphi}( \Tensor(V)_d )
  = \Symm^d(V) \,,
\]
where~$\pi \colon \Tensor(V) \to \Tensor(V)/I$ denotes the canonical projection.
The algebra isomorphism~$\varphi$ is therefore already an isomorphism of graded~{\kalgs}.

Note that if~$(v_i)_{i \in I}$ is a basis of~$V$ then we may further identify the symmetric algebra~$\Symm(V)$ with the poylnomial ring~$k[X_i \suchthat i \in I]$.
Indeed, there exist a unique linear map
\[
          f
  \colon  V
  \to     k[X_i \suchthat i \in I]
\]
with~$f(v_i) = X_i$ for every~$i \in I$, and this linear maps induces by the universal property of the symmetric algebra~$\Symm(V)$ an algebra homomorphism
\[
          \alpha
  \colon  \Symm(V)
  \to     k[X_i \suchthat i \in I]
\]
with~$\alpha(v_i) = X_i$ for every~$i \in I$.
It follows on the other hand from the universal property of the polynomial ring~$k[X_i \suchthat i \in I]$ that there exist a unique algebra homomorphism
\[
          \beta
  \colon  k[X_i \suchthat i \in I]
  \to     \Symm(V)
\]
with~$\beta(X_i) = v_i$ for every~$i \in I$.
As the symmetric algebra~$\Symm(V)$ is generated by the basis elements~$v_i$, and the polynomial ring~$k[X_i \suchthat i \in I]$ is generated by the variables~$X_i$, we find that~$\varphi$ and~$\psi$ are mutually inverse on a set of algebra generators, and are hence mutually inverse.

The isomorphism~$\alpha$ maps for every~$d \geq 0$ the homogeneous component~$\Symm^d(V)$ onto the homogeneous component~$k[X_i \suchthat i \in I]_d$, and is hence already an isomorphism of graded~{\kalgs}.
We therefore find that
\[
        \Tensor(V)/I
  \cong \Symm(V)
  \cong k[X_i \suchthat i \in I]
\]
as graded~{\kalgs}.











\subsection{}

This is just the filtration associated to the grading of the tensor algebra from part~\ref{grading on tensor algebra} of the exercise.





\subsection{}

We won’t give a complete solution to this problem, but instead state the relevant results and sketch how one can proceed to prove them:

Let~$A$ be an associative algebra and for all~$x, y \in A$ let
\[
            [x,y]
  \defined  xy - yx
\]
be the their commutator.
We can then consider the algebra
\[
            U(A)
  \defined  \Tensor(A)/I
\]
where~$I \subseteq \Tensor(A)$ is the {\twosided} ideal given by
\[
            I
  \defined  \genideal{
              x \tensor y - y \tensor x - [x,y]
              \suchthat
              x, y \in A
            } \,.
\]
(The ideal~$I$ is built in precisely such a way that
\[
    [\class{x}, \class{y}]_{(\Tensor(V)/I)}
  = \class{x} \, \class{y} - \class{y} \, \class{x}
  = \class{x \tensor y - y \tensor x}
  = \class{[x,y]_A}
\]
for all~$x, y \in A$, i.e.\ such that the commutator in~$A$ coincides with the one in~$\Tensor(V)/I$.)
We then have the following classical result:

\begin{theorem}[{\pbw}]
  Let~$A$ be a~{\kalg} and let~$(x_i)_{i \in I}$ be a basis of~$A$ where~$(I, \leq)$ is a linearly ordered set.
  Then the ordered monomials
  \[
    x_{i_1}^{n_1} \dotsm x_{i_r}^{n_r}
  \]
  with~$r \geq 0$,~$n_1, \dotsc, n_r \geq 1$ and~$i_1 < \dotsb < i_r$ form a basis of~$U(A)$.
  (Here we write for~$x \in A$ the resulting element~$\class{x} \in U(A)$ by abuse of notation again as~$x$.)
\end{theorem}

\begin{remark}
  The {\pbw} theorem actually holds for every Lie algebra~$\mathfrak{g}$ over the field~$k$, where one defines the \emph{universal enveloping algebra} of~$\mathfrak{g}$ as
  \[
      \mathcal{U}(\mathfrak{g})
    = \Tensor(\mathfrak{g})
      /
      \genideal{
        x \tensor y - y \tensor x - [x,y]
      \suchthat
        x, y \in \mathfrak{g}
      } \,.
  \]
  The universal enveloping algebra~$\mathcal{U}(\mathfrak{g})$ is the \enquote{most general}~{\kalg} which contains~$\mathfrak{g}$ as a Lie subalgebra.
  As a functor,~$\mathcal{U}$ it is left adjoint to the forgetful functor~$\kAlg \to \kLie$.
\end{remark}

The (standard) proof of the {\pbw} theorem proceeds similar to the proof that the monomials~$x^n y^m$ with~$n, m \geq 0$ form a basis of the first Weyl algebra~$A_1 \cong k\gen{x, y}/\genideal{yx - xy - 1}$:%
\footnote{One can in fact derive the result about the Weyl algebra from the {\pbw} theorem.}
\begin{itemize}
  \item
    One first notes that the algebra~$U(A) = \Tensor(A)/I$ is spanned by the residue classes
    \begin{equation}
      \label{first generating set}
      x_{i_1} \dotsm x_{i_n}
      \qquad
      \text{where~$n \geq 0$ and~$i_1, \dotsc, i_n \in I$}
    \end{equation}
    because the tensor algebra~$\Tensor(V)$ has the simple tensors~$x_{i_1} \tensor \dotsb \tensor x_{i_n}$ with~$n \geq 0$ and~$i_1, \dotsc, i_n \in I$ as a basis.
    One can then use the rearrangement rule
    \[
        y x
      = \class{y \tensor x}
      = \class{x \tensor y} + \class{[y,x]}
      = x y + [y,x]
    \]
    for~$x, y \in A$ to show that every monomial of the form~\eqref{first generating set} can be written as a linear combination of the monomials
    \begin{equation}
      \label{second generating set}
      x_{i_1} \dotsm x_{i_n}
      \qquad
      \text{where~$n \geq 0$ and~$i_1, \dotsc, i_n \in I$ with~$i_1 \leq \dotsb \leq i_n$} \,.
    \end{equation}
    (Note that this is precisely the proposed basis from the {\pbw} theorem.)
  \item
    The hard part is to show that the monomials~\eqref{first generating set} are linearly independent.
    For this one uses the trick introduced in the tutorial:
    One constructs an~{\module{$U(A)$}}~$M$ with~{\kbasis}
    \begin{equation}
      \label{actual basis}
      X_{i_1} \dotsm X_{i_r}
      \qquad
      \text{where~$n \geq 0$ and~$i_1, \dotsc, i_n \in I$ with~$i_1 \leq \dotsb \leq i_n$}
    \end{equation}
    such that the action of every~$x_i$ on the basis~\eqref{actual basis} behaves the same as the action of~$x_i$ on~\eqref{second generating set}.
    By using the linear independence of~\eqref{actual basis} it can then be conluded that~\eqref{second generating set} is also linearly independent.
    (The construction of~$M$ is rather complicated, and we will not attempt to do this here.)
\end{itemize}


The above version of the {\pbw} theorem is also known as the \enquote{concrete version}, and it is equivalent to the following \enquote{abstract version}:

\begin{theorem}[{\pbw}]
  Let~$A$ be a~{\kalg} and let~$\pi \colon \Tensor(A) \to U(A)$ be the canonical projection (which is a homomorphism of filtered~{\kalg}).
  Then the homomorphisms of graded~{\kalgs}
  \[
            \gr(\pi)
    \colon  \Tensor(A)
    \to     \gr(U(A))
    \qquad\text{and}\qquad
        \Tensor(A)
    \to \Symm(A)
  \]
  have the same kernel, and hence induce an isomorphism of~{\kalgs}
  \[
          \gr(U(A))
    \cong \Symm(A) \,.
  \]
\end{theorem}

One can then also further identify the symmetric algebra~$\Symm(A)$ with a polynomial ring~$k[X_i \suchthat i \in I]$ (as graded~{\kalgs}) by choosing a basis~$(a_i)_{i \in I}$ of~$A$.
It follows in particular for~$A = \mat{n}{k}$ that
\[
        \gr(U(\mat{n}{k}))
  \cong \Symm(\mat{n}{k})
  \cong k[X_1, \dotsc, X_{n^2}] \,.
\]














