\section{}

We instead show the following lemma, from which we then deduce the exercise:

\begin{lemma}
  \label{bourbaki lemma}
  Let~$A = \bigoplus_{d \geq 0} A_d$ be a graded, commutative~{\kalg}.
  Then for a family~$(x_i)_{i \in I}$ of homogenenous elements~$x_i$ of degree~$\deg(x_i) \geq 1$ the following conditions are equivalent:
  \begin{enumerate}
    \item
      \label{ideal}
      The elements~$(x_i)_{i \in I}$ generate the ideal~$A^+ \defined \bigoplus_{d \geq 1} A_d$ of~$A$.
    \item
      \label{algebra}
      The elements~$(x_i)_{i \in I}$ generate~$A$ as an~\dash{$A_0$}{algebra}.
  \end{enumerate}
\end{lemma}

\begin{proof}
  Suppose that the ideal~$A^+$ is generated by~$(x_i)_{i \in I}$ and let~$A' \defined A_0[x_i \suchthat i \in I]$ be the~\dash{$A_0$}{subalgebra} of~$A$ generated by~$(x_i)_{i \in I}$.
  We show by induction on~$d \geq 0$ that~$A_d \subseteq A'_d$.
  
  For~$d = 0$ we have~$A_0 \subseteq A'$ by definition of~$A'$.
  Suppose that~$d \geq 1$ and that~$A_0, \dotsc, A_{d-1} \in A'$.
  It then holds that~$A_d \subseteq A^+$ and hence for~$x \in A_d$ that~$x \in A^+$.
  It follows that
  \[
      x
    = \sum_{i \in I} a_i x_i
  \]
  for some coefficients~$a_i \in A$ (where~$a_i = 0$ for all but finitely many~$i$).
  We can decompose every coefficient~$a_i$ as~$a_i = \sum_{d' \geq 0} a'_{i,d'}$ with~$a'_{i,d'}$ homogeneous of degree~$d'$.
  We then have that
  \[
      x
    = \sum_{i \in I} a_i x_i
    = \sum_{d' \geq 0} \sum_{i \in I} a'_{i,d'} x_i \,.
  \]
  The element~$x$ is homogeneous of degree~$n$, so we find by comparing the~\dash{$d$}{coefficients} of both sides that
  \[
      x
    = \sum_{i \in I} a'_{i, d-\deg(x_i)} x_i
  \]
  where we set~$a'_{i,d'} = 0$ for~$d' < 0$.
  We thus find that by replacing~$a_i$ with~$a'_{i,d-\deg(x_i)}$ we may assume that the coefficients~$a_i$ are homogeneous of degree~$d - \deg(x_i) < d - 1$.
  It then follows by induction hypothesis that the coefficients~$a_i$ are contained in~$A'$.
  It then follows that
  \[
              x
    =         \sum_{i \in I} a_i x_i
    \in       \sum_{i \in I} A' x_i
    \subseteq \sum_{i \in I} A'
    =         A' \,.
  \]
  This shows that~$A_d \subseteq A'$.
  
  Suppose on the other and that~$A_0$ is generated by~$(x_i)_{i \in I}$ as an~\dash{$A_0$}{algebra}.
  Then the monomials~$x_{i_1} \dotsm x_{i_r}$ with~$i_1, \dotsc, i_r \in I$ generate~$A$ as an~{\module{$A$}}.
  These generators are homogeneous, and hence the submodule~$A^+$ is generated by the monomials with~$r \geq 1$.
  We have for these monomials that
  \[
        x_{i_1} \dotsm x_{i_r}
    =   x_{i_1} \dotsm x_{i_{r-1}} x_{i_r}
    \in A x_{i_r} \,.
  \]
  This shows that~$A^+$ is generated by~$(x_i)_{i \in I}$ as an ideal.
\end{proof}

We now show how the exercise follows from the above \lcnamecref{bourbaki lemma}:

If~$A^+$ is generated by finitely many elements~$x_1, \dotsc, x_n$, then we may replace~$x_i$ by all of its homogeneous components to assume that~$x_1, \dotsc, x_n$ are homogeneous.
It then follows from the above \lcnamecref{bourbaki lemma} that~$A$ is generated by~$x_1, \dotsc, x_n$ as an~\dash{$A_0$}{algebra}.

Suppose on the other hand~$A$ is generated by elements~$x_1, \dotsc, x_n$ as an~\dash{$A_0$}{algebra}
We may replace every generator~$x_i$ by all of its homogeneous components to assume that~$x_1, \dotsc, x_n$ are homogeneous.
We may also remove all generators of degree~$0$, as they are not needed.
It then follows from the above \lcnamecref{bourbaki lemma} that~$x_1, \dotsc, x_n$ generate~$A^+$ as an ideal.

\begin{remark}
  If it not needed that~$A_0$ is noetherian.
  But if~$A_0$ is notherian, then one can given a shorter proof of one of the impliciation:
  
  If~$A_0$ is noetherian and~$A$ is finitely generated as an~\dash{$A_0$}{algebra} then~$A$ is also noetherian by Hilbert’s~basis~theorem.
  It follows that the ideal~$A^+$ of~$A$ is finitely generated.
\end{remark}




