\section{}





\subsection*{A missing proof}

We will give two proof of the following lemma from linear algebra which we used in the tutorial:

\begin{lemma}
  Let~$M$ be an~{\module{$R$}} and let~$P \subseteq N \subseteq M$ be submodules.
  If a submodule~$C \subseteq M$ is a direct complement of~$P$ in~$M$, then~$C \cap N$ is a direct complement of~$P$ in~$N$, i.e.\ if~$M = P \oplus C$ then~$N = P \oplus (C \cap N)$.
\end{lemma}



\subsubsection*{First Proof}

For the first proof we use linear algebra.

\begin{recall}
  \label{modularity}
  Let~$M$ be an~{\module{$R$}} and let~$P \subseteq N \subseteq M$ be submodules.
  Then
  \[
      P + (C \cap N)
    = (P + C) \cap N
  \]
  for every submodule~$C \subseteq M$.
\end{recall}

\begin{remark}
  \Cref{modularity} states that the lattice of submodules of~$M$ is modular.
\end{remark}

\begin{proof}[First proof]
  It holds that
  \[
      P \cap (C \cap N)
    = P \cap C \cap N
    = 0 \cap N
    = 0 \,,
  \]
  and it holds by~\cref{modularity} that
  \[
      P + (C \cap N)
    = (P + C) \cap N
    = M \cap N
    = N \,.
  \]
  Hence~$N = P \oplus (C \cap N)$.
\end{proof}



\subsubsection*{Second Proof}

For the second proof we will also use some linear algebra:

\begin{definition}
  If~$X$ is a set then a map~$e \colon X \to X$ is \emph{idempotent} if~$e^2 = e$.
\end{definition}

\begin{definition}
  Let~$M$ be an~{\module{$R$}} and let~$P, C \subseteq M$ be two submodules for which~$M = P \oplus C$.
  Then the map~$e \colon M \to M$ given by
  \[
      e(p + c)
    = p
  \]
  for all~$p \in P$ and~$c \in C$ is the \emph{projection} onto~$P$ along(side)~$C$.
\end{definition}


\begin{recall}
  Let~$M$ be an~{\module{$R$}}.
  \begin{enumerate}
    \item
      Let~$P, C \subseteq M$ be submodules with~$M = P \oplus C$.
      Then the projection~$e_{P,C} \colon M \to M$ onto~$P$ along~$C$ is an idempotent endomorphisms with~$\im(e) = P$ and~$\ker(e) = C$.
    \item
      If~$e \colon M \to M$ is any idempotent endomorphism then
      \[
        M = \im(e) \oplus \ker(e) \,.
      \]
    \item
      The above constructions yield a~{\onetoone} correspondence
      \begin{align*}
          \left\{
            (P,C)
          \suchthat*
            \begin{tabular}{c}
              submodules  \\
              $P, C \subseteq M$  \\
              with~$M = P \oplus C$
            \end{tabular}
          \right\}
        &\longleftrightarrow
          \left\{
            \begin{tabular}{c}
              idempotent  \\
              endomorphisms \\
              $e \colon M \to M$
            \end{tabular}
          \right\} \,,
          \\
            (P,C)
          &\longmapsto
            e_{P,C} \,,
          \\
            (\im(e), \ker(e))
          &\longmapsfrom
            e \,.
      \end{align*}
  \end{enumerate}
\end{recall}


\begin{proof}[Second proof]
  Let~$e \colon M \to M$ be the projection onto~$P$ along~$C$.
  The image of~$e$ is the submodule~$P$ of~$N$, and hence we can restrict the endomorphism~$e$ to an endomorphism~$e' \colon N \to N$.
  This restriction~$e'$ is again idempotent and therefore leads to a direct sum decomposition
  \[
      N
    = \im(e') \oplus \ker(e') \,.
  \]
  The first summand is given by~$e'(N) = P$.
  Indeed, the inclusion~$e'(N) \subseteq P$ holds by construction of~$e$, and the inclusion~$P \subseteq e'(N)$ follows from~$P \subseteq N$.
  The second summand is given by~$\ker(e') = \ker(e) \cap N = C \cap N$.
  We thus find that~$N = P \oplus (C \cap N)$, as desired.
\end{proof}





\subsection*{A Remark Regarding Nakayama}

We have seen the following \lcnamecref{fg has maximal submodules} in the tutorial.

\begin{lemma}
  \label{fg has maximal submodules}
  Every nonzero fininitely generated~{\module{$R$}}~$M$ contains a maximal submodule.
\end{lemma}

We give a short proof Nakayama’s~lemma which is based on this \lcnamecref{fg has maximal submodules}.

\begin{definition}
  \leavevmode
  \begin{enumerate}
    \item
      The \emph{radical} of an~{\module{$R$}}~$M$ is the intersection of all of its maximal submodules;
      it is denoted by~$\rad(M)$.
    \item
      The \emph{Jacobson~radical} of~$M$ is~$\Jac(R) \defined \rad(R)$.
  \end{enumerate}
\end{definition}

\begin{remark}
  One may call~$\Jac(R)$ the \emph{left Jacobson~radical} of~$R$, as it is the intersection of all maximal left%
  \footnote{Because for us, \enquote{module} means \enquote{left module}.}
  ideals of~$R$.
  One can then also define the \emph{right Jacobson~radical} of~$R$.
  But it turns out that both notions of Jacobson radical coincide.
  We will not need this here and will therefore not prove this.%
  \footnote{The author also doesn’t know a good proof. of this}
\end{remark}

We get from \cref{fg has maximal submodules} the following corollary:

\begin{corollary}
  \label{radical is proper}
  If~$M$ is a nonzero finitely generated~{\module{$R$}} then its radical~$\rad(M)$ is a proper submodule of~$M$.
\end{corollary}

The radical of a module is functorial in the following sense:

\begin{lemma}
  \label{functoriality of radical}
  Let~$M$ and~$N$ be~{\modules{$R$}} and lets~$f \colon M \to N$ be a homomorphism of~{\modules{$R$}}.
  Then~$f(\rad(M)) \subseteq \rad(N)$.
\end{lemma}

\begin{proof}
  If~$P \subseteq N$ is any maximal submodule then its preimage~$P' \defined f^{-1}(P)$ is the kernel of the composition~$M \xto{f} N \to N/P$.
  The homomorphism~$f$ therefore induces an injective homomorphism
  \[
                M/P'
    \inclusion  N/P \,.
  \]
  The quotient module~$N/P$ is simple because~$P$ is maximal.
  It follows that~$M/P' = 0$ or~$M/P' \cong N/P$.
  The submodule~$P'$ is therefore either the whole of~$M$ or maximal in~$M$.
  In both cases the radical~$\rad(M)$ is contained in~$P'$.
  
  This shows that~$\rad(M) \subseteq f^{-1}(P)$ for every maximal submodule~$P \subseteq N$, and hence that~$f(\rad(M)) \subseteq P$.
  It follows that~$f(\rad(M)) \subseteq \rad(N)$.
\end{proof}

\begin{corollary}
  \label{jacobson to radical}
  Let~$M$ be an~{\module{$R$}}.
  Then~$\Jac(R) M \subseteq \rad(M)$.
\end{corollary}

\begin{proof}
  Consider for~$x \in M$ the map
  \[
            \rho_x
    \colon  R
    \to     M \,,
    \quad   r
    \mapsto rx \,.
  \]
  The map~$\rho_x$ is a homomorphism of~{\modules{$R$}}, and so it follows that
  \[
              \Jac(R) x
    =         \rho_x( \Jac(R) )
    =         \rho_x( \rad(R) )
    \subseteq \rad(M)
  \]
  by \cref{functoriality of radical}.
\end{proof}

We are now ready to state and prove Nakayama’s~lemma.

\begin{lemma}[Nakayama]
  If~$M$ is a finitely generated~{\module{$R$}} with~$\Jac(R) M = M$ then~$M = 0$.
\end{lemma}

\begin{proof}
  If~$M$ were nonzero then it would follow from \cref{radical is proper} and \cref{jacobson to radical} that~$\Jac(R) M \subseteq \rad(M) \subsetneq M$.
\end{proof}





