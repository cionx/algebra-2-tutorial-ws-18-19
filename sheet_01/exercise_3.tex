\section{}





\subsection{}


\begin{remark}
  The proven isomorphism~$k[G] \cong k[x]/(x^{p^r})$ depends heavily on the fact that~$\ringchar(k) = p$:
  The isomorphism
  \[
          k[G]
    \cong k[\Integer/p^r]
    \cong k[x](x^{p^r} - 1)
  \]
  holds for every field~$k$, while the further isomorphism
  \[
          k[x]/(x^{p^r} - 1)
    \cong k[x](x^{p^r})
  \]
  uses that~$\ringchar(k) = p$ to transform
  \[
      x^{p^r} - 1
    = x^{p^r} - 1^{p^r}
    = (x - 1)^{p^r} \,.
  \]
  If for example~$k = \Complex$ instead, then
  \[
      x^{p^r} - 1
    = (x - \omega_1) \dotsm (x - \omega_{p^r})
  \]
  for the~\dash{$p^r$} {th} roots of unity~$\omega_1, \dotsc, \omega_{p^r} \in \Complex$ given by~$\omega_j = e^{2 \pi i j / p^r}$.
  These roots of unity~$\omega_1, \dotsc, \omega_{p^r}$ are pairwise distinct and so it follows from the chinese reminder theorem that
  \begin{align*}
            \Complex[G]
    \cong  \Complex[x]/(x^{p^r} - 1)
    &\cong  \Complex[x]( (x - \omega_1) \dotsm (x - \omega_{p^r}) ) \\
    &\cong  \prod_{j=1}^{p^r} \Complex[x]/(x - \omega_j)
    \cong  \prod_{j=1}^{p^r} \Complex
    =      \Complex^{\times p^r} \,.
  \end{align*}
\end{remark}







\subsection{}

We have seen in the tutorial how one can use the first part of the exercise to classify the {\fd} indecomposable, resp.\ irreducible representations of~$G$ over~$k$.
But one can also classify these representations by instead using linear algebra, effectively ignoring the first part of the exercise:

Let~$V$ be a representation of~$G$ over~$k$.
Then the cyclic generator~$g \in G$ acts on~$V$ via an endomorphisms
\[
          \varphi
  \colon  V
  \to     V \,,
  \quad   v
  \mapsto g.v \,.
\]
It follows from~$g^{p^r} = 1$ that~$\varphi^{p^r} = \id$, and hence
\[
    0
  = \varphi^{p^r} - \id
  = \varphi^{p^r} - \id^{p^r}
  = (\varphi - \id)^{p^r}
  = p(\varphi)
\]
for the polynomial~$p(t) \defined (t-1)^{p^r} \in k[t]$, where we used that~$\ringchar(k) = p$.
If the vector space~$V$ is {\fd}\footnote{We don’t actually need this assumption.} then this shows that~$\varphi$ is triangularizable with~$1$ as its only eigenvalue.
We can therefore consider its Jordan normal form, which is with respect to a suitable basis~$B = (b_1, \dotsc, b_n)$ of~$V$ given by a block diagonal matrix
\[
    J
  = \begin{bmatrix}
      J_1 &         &       \\
          & \ddots  &       \\
          &         & J_s
    \end{bmatrix}
\]
with (unipotent) Jordan blocks
\[
    J_i
  = \begin{bmatrix}
      1 &       1 &         &   \\
        & \ddots  & \ddots  &   \\
        &         & \ddots  & 1 \\
        &         &         & 1
    \end{bmatrix}
  \in \mat{n_i}{k} \,.
\]
The decomposition of the matrix~$J$ as a block diagonal matrix corresponds to a decomposition of the vector space~$V$ into~\dash{$\varphi$}{invariant} subspaces.
More precisely, we set
\begin{align*}
            U_1
  &\defined \gen{b_1, \dotsc, b_{n_1}}_k  \,, \\
            U_2
  &\defined \gen{b_{n_1 + 1}, \dotsc, b_{n_1 + n_2}}_k \,,  \\
  &\;\;\,\vdots  \\
            U_s
  &\defined \gen{b_{n_1 + \dotsb + n_{s-1} + 1}, \dotsc, b_n}_k \,.
\end{align*}
Then~$V = U_1 \oplus \dotsb \oplus U_s$ is a decomposition into~\dash{$\varphi$}{invariant} subspaces, and hence a decomposition into subrepresentations.

For~$V$ to be indecomposable we therefore need that~$s = 1$.
We have thus shown that for every {\fd} indecomposable representation of~$G$ over~$k$ there exists a basis of~$V$ with respect to which the action of the cylic generator~$g \in G$ is given by the matrix
\[
  U_n
  \defined
  \begin{bmatrix}
    1 &       1 &         &   \\
      & \ddots  & \ddots  &   \\
      &         & \ddots  & 1 \\
      &         &         & 1
  \end{bmatrix}
  \in \mat{n}{k} \,,
\]
where~$n = \dim(V)$.
This observation results in a classification of all~{\fd} indecomposable, resp.\ irreducible representations of~$G$ over~$k$:


\begin{proposition}
  \leavevmode
  \begin{enumerate}
    \item
      For every~$n = 1, \dotsc, p^r$ there exists a unique linear action of~$G$ on~$k^n$ for which the cyclic generator~$g \in G$ acts by multiplication with the matrix~$U_n$.
  \end{enumerate}
  We denote the resulting~\dash{$n$}{dimensional} representation of~$G$ by~$V_n$.
  \begin{enumerate}[resume]
    \item
      \label{classification of subrep}
      The subrepresentations of~$V_n$ are precisely~$W_i \defined \gen{e_1, \dotsc, e_i}_k$ for~$i = 0, \dotsc, n$.
    \item
      The representation~$V_n$ is for every~$n = 1, \dotsc, p^r$ indecomposable.
      The representation~$V_n$ is irreducible if and only if~$n = 1$.
    \item
      The indecomposable representations~$V_1, \dotsc, V_{p^r}$ are a set of representatives for the isomorphism classes of {\fd} indecomposable representations of~$G$ over~$k$.
      The representation~$V_1$ is up to isomorphism the only irreducible representation of~$G$ over~$k$.
  \end{enumerate}
\end{proposition}


\begin{proof}
  \leavevmode
  \begin{enumerate}
    \item
      We need to show that there exists a unique group homomorphism~$\rho \colon G \to \GL_n(k)$ with~$\varphi(g) = U_n$.
      The uniqueness follows from~$G$ being generated by~$g$.
      For the existence we only need to check that~$U_n^{p^r} = \Id$.
      This holds, because if we write
      \begin{gather*}
          U_n
        = \Id + N_n
      \shortintertext{with}
            N_n
        =   \begin{bmatrix}
              0 & 1       &         &   \\
                & \ddots  & \ddots  &   \\
                &         & \ddots  & 1 \\
                &         &         & 0
            \end{bmatrix}
        \in \mat{n}{k}
      \end{gather*}
      then the matrices~$\Id$ and~$N_n$ commute, and it follows that
      \[
          U_n^{p^r}
        = (\Id + N_n)^{p^r}
        = \Id^{p^r} + N_n^{p^r}
        = \Id + 0
        = \Id \,.
      \]
      
    \item
      We abbreviate~$N \defined N_n$.
      The subrepresentations of~$V_n$ are precisely the~\dash{$U_n$}{invariant} subspaces, which are by the decomposition~$U_n = \Id + N$ precisely the~\dash{$N$}{invariant} subspaces (because every subspace is~\dash{$\Id$}{invariant}).
      The subspaces~$W_i$ are~\dash{$N$}{invariant} because~$N e_j = e_{j-1}$.
      
      We need to show that every \dash{$N$}{invariant} subspaces~$W \subseteq V$ is of the form~$W = W_i$ for some~$i$.
      It sufficies to show that for every~$x \in V$ the generated~\dash{$N$}{invariant} subspace
      \[
          \gen{ x }
        = \gen{ x, N x, N^2 x, \dotsc }_k
      \]
      is of the form~$W_i$ for some~$i$;
      it then follows that~$W = \sum_{x \in W} \gen{x}$ is again of the form~$W_i$ for some~$i$, because every sum of~$W_j$’s is again a~$W_i$.
      
      If
      \[
          x
        = \vect{x_1 \\ \vdots \\ x_i \\ 0 \\ \vdots \\ 0 }
      \]
      with~$x_i \neq 0$ then~$N^j x \subseteq W_i$ for every~$j \geq 0$ and hence~$\gen{x} \subseteq W_i$.
      We have on the other hand that
      \[
        x
        = \vect{x_1 \\ \vdots \\ x_{i-1} \\ x_i \\ 0 \\ \vdots \\ 0 } \,,
        \qquad
        N x
        = \vect{x_2 \\ \vdots \\ x_i \\ 0 \\ 0 \\ \vdots \\ 0 } \,,
        \qquad
        \dotsc,
        \qquad
        N^{i-1} x
        = \vect{x_i \\ 0 \\ \vdots \\ \vdots \\ \vdots \\ 0}
      \]
      are linearly independent, and hence
      \[
              \dim \gen{x}
        \geq  i
        =     \dim W_i \,.
      \]
      Together this shows that~$\gen{x} = W_i$.
      
    \item
      This follows from part~\ref{classification of subrep}.
      
    \item
      We have seen previously that every {\fd} indepecomposable representation of~$G$ over~$k$ is isomorphic to some~$V_n$.
      The representations~$V_1, \dotsc, V_{p^r}$ are pairwise \dash{non}{isomorphic} because they have different dimensions.
    \qedhere
  \end{enumerate}
\end{proof}



