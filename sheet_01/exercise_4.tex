\section{}

Before we begin with the exercise itself, we recall some group theory:

Let~$G$ be a group and let~$H \subseteq G$ be a subgroup.
Then~$G$ acts on the set of left cosets~$X \defined G/H = \{ gH \suchthat g \in G\}$ via
\[
    g \cdot (g' H)
  = (g g') H
\]
for all~$g \in G$ and~$g'H \in X$.
This action of~$G$ on~$X$ is transitive and the stabilizer of~$H = 1H \in X$ is given by~$\Stab_G(H) = H$.
Moreover, the map
\[
          G
  \to     X \,,
  \quad   g
  \mapsto gx
\]
is~\dash{$G$}{equivariant} for every~$x \in X$.
The Orbit-Stabilizer theorem asserts that every transitive~\dash{$G$}{set} looks like this:

\begin{theorem}[Orbit-Stabilizer theorem]
  Let~$G$ be a group and let~$X$ be a~\dash{$G$}{set}.
  Then for every~$x \in X$ the map
  \[
            G/G_x
    \to     G.x \,,
    \quad   gG_x
    \mapsto g.x
  \]
  is both bijective and~\dash{$G$}{equivariant}, i.e.\ it is an isomorphism of~\dash{$G$}{sets}.
\end{theorem}

The Orbit-Stabilizer theorem asserts that if we want for some group~$G$, subgroup~$H \subseteq G$ and set~$X$ construct a bijection~$G/H \to X$, then we need to construct a transitive action of~$G$ on~$X$ such that~$H$ is the stabilizer of some point~$x \in X$.

\begin{example}[Topology]
  The group~$\SO(n+1)$ acts on the sphere~$\sphere^n$ by rotation.
  This action is transitive, and the stabilizer of~$e_1 \in \sphere^n$ is given by~$\SO(n)$, when regarded as a subgroup of~$\SO(n+1)$.
  It hence follows from the Orbit-Stabilizer theorem that
  \[
          \SO(n+1)/\SO(n)
    \cong \sphere^n
  \]
  as~\dash{$\SO(n+1)$}{sets}.
  (It follows from~$\SO(n+1)$ being compact and~$\sphere^n$ being Hausdorff that this is already a homeomorphism.)
\end{example}





\subsection{}

We consider the action of~$G = \GL_n(k)$ on the set~$X$ of flags in~$k^n$ via
\[
    g . (F_i)_{i=0}^n
  = ( gF_i )_{i=0}^n
\]
where for every subspace~$U \subseteq k^n$ and every~$g \in G$ the subspace~$gU \subseteq k^n$ is given by
\[
    gU
  = \{ g u \suchthat u \in U \} \,.
\]
This does indeed define an action of~$G$ on~$X$ because
\[
    \Id.(F_i)_{i=0}^n
  = (\Id F_i)_{i=0}^n
  = (F_i)_{i=0}^n
\]
for every~$(F_i)_{i=0}^n \in X$, and
\[
    g.(h.(F_i)_{i=0}^n)
  = g.(h F_i)_{i=0}^n
  = (g h F_i)_{i=0}^n
  = (gh).(F_i)_{i=0}^n
\]
for all~$g, h \in G$ and all~$(F_i)_{i=0}^n \in X$.

We now consider the standard flag~$S \in X$ given by
\[
            S_i
  \defined  \gen{e_1, \dotsc, e_i}_k
\]
for all~$i = 0, \dotsc, n$.
We have for every matrix~$g \in G$ that
\begingroup
\allowdisplaybreaks
\begin{align*}
      {}& g \in \Stab_G(S) \\
  \iff{}& g.S = S \\
  \iff{}& \text{$g.\gen{e_1, \dotsc, e_i}_k = \gen{e_1, \dotsc, e_i}_k$ for every~$i = 0, \dotsc, n$} \\
  \iff{}& \text{$g.\gen{e_1, \dotsc, e_i}_k \subseteq \gen{e_1, \dotsc, e_i}_k$ for every~$i = 0, \dotsc, n$} \\
  \iff{}& \text{$g.e_j \in \gen{e_1, \dotsc, e_i}_k$ for all~$1 \leq j \leq i \leq n$} \\
  \iff{}& \text{the~\dash{$j$}{th} column of~$g$ is contained in~$\gen{e_1, \dotsc, e_i}_k$ for all~$1 \leq j \leq i \leq n$}  \\
  \iff{}& \text{$g$ is of the form
            $
            \begin{bmatrix}
              * & \cdots  & *       \\
                & \ddots  & \vdots  \\
                &         & *
             \end{bmatrix}
            $
          } \\
  \iff{}& \text{$g$ is upper triangular} \,,
\end{align*}
\endgroup
which shows that
\[
    \Stab_G(S)
  = B \,.
\]

We have that~$G.S = X$:
There exists for every flag~$(F_i)_{i=0}^n \in X$ a basis~$b_1, \dotsc, b_n$ of~$k^n$ such that~$b_1, \dotsc, b_i$ is for every~$i = 0, \dotsc, n$ a basis of~$F_i$.
The matrix~$g = (b_1, \dotsc, b_n)$ (i.e.\ the matrix whose columns are the basis vectors~$b_1, \dotsc, b_n$) is then invertible with
\[
    g.S_i
  = g.\gen{e_1, \dotsc, e_i}_k
  = \gen{g.e_1, \dotsc, g.e_i}_k
  = \gen{b_1, \dotsc, b_i}_k
  = F_i
\]
for every~$i = 0, \dotsc, n$, and hence~$g.S = X$.

It now follows from the Orbit-Stabilizer theorem that we have a bijection
\[
          \varphi
  \colon  G/B
  \to     X \,,
  \quad   gB
  \mapsto g.S
\]
which is also~\dash{$G$}{equivariant}.
The bijection~$\varphi$ assigns to each matrix~$g \in G$ with columns~$g_1, \dotsc, g_n$ the flag~$g.S = (F_i)_{i=0}^n$ with~$F_i = \gen{g_1, \dotsc, g_i}_k$ for every~$i = 0, \dotsc, n$.





\subsection{}

The bijection~$\varphi$ is a~\dash{$G$}{equivariant}, and therefore also~$B$ equivariant, and thus induces a bijection
\[
          \{\text{$B$ orbits in~$G/B$}\}
  \longto \{\text{$B$ orbits in~$X$}\} \,.
\]
In the case~$n = 2$ every flag~$(F_i)_{i=0}^2 \in X$ is uniquely determined by its middle term~$F_1$ (because~$F_0 = 0$ and~$F_2 = k^2$), which is an arbitrary \dash{one}{dimensional} subspace of~$k^2$.
We thus have a bijection
\[
            \psi
  \colon    X
  \to       \{ \text{\dash{one}{dimensional} subspaces of~$k^2$} \}
  \defines  X'  \,,
  \quad     (F_i)_{i=0}^2
  \mapsto   F_1 \,,
\]
and this bijection is~\dash{$G$}{equivariant}, and hence also~\dash{$B$}{equivariant}.
We therefore get another induced bijection
\[
          \{\text{$B$ orbits in~$X$}\}
  \longto \{\text{$B$ orbits in~$X'$}\} \,.
\]
Suppose now that~$L \in X'$ is a \dash{one}{dimensional} subspace of~$k^2$.
Then~$L$ is spanned by a single nonzero vector
\[
      \vect{x \\ y}
  \in L \,.
\]
For~$h \in B$ with~$h = \begin{bsmallmatrix} a & b \\ {} & c \end{bsmallmatrix}$ the space~$h.L$ is spanned by the single vector
\begin{equation}
  \label{action of triangle on vector}
  \begin{bmatrix}
    a   & b \\
    {}  & c
  \end{bmatrix}
  \vect{x \\ y}
  =
  \vect{ax + by \\ cy} \,.
\end{equation}
We can now distinguish between two cases:
\begin{itemize}
  \item
    If~$y \neq 0$ then we can choose ~$c = 1/y$,~$a = 1$ and~$b = -x/y$ to find that the~\dash{$B$}{orbit} of~$L$ contains the line~$\gen{e_1}$.
  \item
    If~$y = 0$ then~$x \neq 0$ and we can choose~$c = 1$,~$a = 1/x$ and~$b = 0$ to find that the~\dash{$B$}{orbit} of~$L$ contains the line~$\gen{e_2}$.
\end{itemize}
If follows from~\eqref{action of triangle on vector} that the~\dash{$B$}{orbit} of~$\gen{e_2}$ does not contain any line~$L'$ with~$e_1 \in L'$ (because it follows from~$c, y \neq 0$ that~$cy \neq 0$).
Hence the lines~$\gen{e_1}$ and~$\gen{e_2}$ are not in the same~\dash{$B$}{orbit}.

This shows altogether that~$\gen{e_1}$ and~$\gen{e_2}$ form a set of representatives for the~\dash{$B$}{orbits} on~$X'$.
This shows that there exist precisely two~\dash{$B$}{orbits} on~$X'$, and hence two~\dash{$B$}{orbits} on~$X$.
The two orbits have as a set of representatives the two flags
\begin{equation}
  \label{flag representatives}
            0
  \subseteq \gen{e_1}
  \subseteq k^2 \,,
  \qquad
            0
  \subseteq \gen{e_2}
  \subseteq k^2 \,.
\end{equation}





\subsection{}

It holds for every~$h \in B$ that~$h.\gen{e_1} = \gen{e_1}$.
The orbit of the flag~$F^1 = (0 \subseteq \gen{e_1} \subseteq k^2)$ consists therefore of only~$F^1$ itself.
It follows from
\[
    |X|
  = |G/B|
  = \frac{|G|}{|B|}
  = \frac{(q^2 - 1)(q^2 - q)}{(q-1)^2 q}
  = \frac{q(q+1)(q-1)^2}{(q-1)^2 q}
  = q+1
\]
that the other orbit, namely the one of~$F^2 = (0 \subseteq \gen{e_2} \subseteq k^2)$, consists of~$q$ elements.





\subsection{}

The group~$B$ acts on~$G$ from the left by left multiplication, and from the right by right multiplication.
The space~$H_q$ consists of all functions~$f \colon G \to \Complex$ which are both left invariant and right invariant under these actions of~$B$.
These are precisely those functions~$f \colon G \to \Complex$ which are constants on the double cosets
\[
    BgB
  = \{
      b_1 g b_2
    \suchthat
      b_1, b_2 \in B
    \}
\]
for~$g \in G$.
For the set of double cosets
\[
    B \backslash G / B
  = \{
      B g B
    \suchthat
      g \in G
    \}
\]
a basis of~$H_q$ is therefore given by the characteristic functions~$I_D$ of the double cosets~$D \in B \backslash G / B$.
We can determine~$B \backslash G / B$ with the following lemma:

\begin{lemma}
  \label{double cosets}
  Let~$G$ be a group and let~$K, H \subseteq G$ be two subgroups.
  Then the map
  \[
            H \backslash (G / K)
    \to     H \backslash G / K \,,
    \quad   H(gK)
    \mapsto HgK
  \]
  is a {\welldef} bijection.
\end{lemma}

\begin{proof}
  We start with the {\welldef} map
  \[
            \varphi''
    \colon  G
    \to     H \backslash G / K
    \quad   g
    \mapsto H g K \,.
  \]
  It holds for all~$g \in g$ an~$k \in K$ that
  \[
      \varphi''(g \cdot k)
    = H g k K
    = H g K
    = \varphi''(g) \,,
  \]
  which shows that~$\varphi''$ descends to a {\welldef} map
  \[
            \varphi'
    \colon  G / K
    \to     H \backslash G / K
    \quad   gK
    \mapsto H g K \,.
  \]
  It holds for all~$h \in H$ and~$gK \in G/K$ that
  \[
      \varphi'(h \cdot gK)
    = H hg K
    = H g K
    = \varphi'(gK) \,,
  \]
  which shows that~$\varphi'$ further descends to a {\welldef} map
  \[
            \varphi
    \colon  H \backslash (G / K)
    \to     H \backslash G / K
    \quad   H(gK)
    \mapsto H g K \,.
  \]
  We find in the same way that the {\welldef} map
  \[
            \psi'
    \colon  G
    \to     H \backslash (G / K) \,,
    \quad   g
    \mapsto H(gK)
  \]
  satisfies
  \[
      \psi'(h \cdot g \cdot k)
    = H((hgk)K)
    = H(hgkK)
    = H(hgK)
    = (Hh)(gK)
    = HgK
    = \psi'(g)  \,,
  \]
  and therefore descends to a {\welldef} map
  \[
            \psi'
    \colon  H \backslash G / K
    \to     H \backslash (G / K) \,,
    \quad   KgK
    \mapsto H(gK) \,.
  \]
  The maps~$\varphi$ and~$\psi$ are inverse to each other, which shows that~$\varphi$ is a bijection.
\end{proof}


We have already seen that~$B \backslash (G / B)$ has (for~$n = 2$) precisely two elements.
It follows from \cref{double cosets} that~$B \backslash G / B$ has precisely two elements.
One of those is $B 1 B = B$, the other one is the complement~$B' \defined G \smallsetminus B$ (because~$G$ is the disjoint union of the double cosets in~$B \backslash G / B$).

The space~$H_q$ has therefore two basis elements, namely the characteristic functions~$I_B$ and~$I_{B'}$.

To determine the algebra structure on~$H_q$ we first check that the given multiplication is both {\welldef} and associative.

We have for all~$f_1, f_2 \in H_q$ that
\begin{align*}
      (f_1 \cdot f_2)(bg)
  &=  \frac{1}{\card{B}} \sum_{y \in G} f_1(y) f_2(y^{-1} b g)  \\
  &=  \frac{1}{\card{B}} \sum_{y \in G} f_1(by) f_2((by)^{-1} bg) \\
  &=  \frac{1}{\card{B}} \sum_{y \in G} f_1(by) f_2(y^{-1} b^{-1} bg) \\
  &=  \frac{1}{\card{B}} \sum_{y \in G} f_1(by) f_2(y^{-1} g) \\
  &=  \frac{1}{\card{B}} \sum_{y \in G} f_1(y) f_2(y^{-1} g)  \\
  &=  (f_1 \cdot f_2)(g) \,,
\end{align*}
and
\[
    (f_1 \cdot f_2)(gb)
  = \frac{1}{\card{B}} \sum_{y \in G} f_1(y) f_2(y^{-1} g b)
  = \frac{1}{\card{B}} \sum_{y \in G} f_1(y) f_2(y^{-1} g)
  = (f_1 \cdot f_2)(g)
\]
for all~$g \in G$ and~$b u/n B$, which shows that again~$f_1 \cdot f_2 \in H_q$.
To show that the multiplication is associative we first note that for~$f_1 \cdot f_2 \in H_q$ the product can also be written as
\[
    (f_1 \cdot fv2)(g)
  = \frac{1}{\card{B}} \sum_{y \in G} f_1(y) f_2(y^{-1} b g)
  = \frac{1}{\card{B}} \sum_{\substack{y, y' \in G \\ y y' = g}} f_1(y) f_2(y') \,.
\]
With this we find for all~$f_1, f_2, f_3 \in H_q$ that
\begin{align*}
      (f_1 \cdot (f_2 \cdot f_3)(g)
  &=  \frac{1}{\card{B}}
      \sum_{\substack{y, y' \in G \\ y y' = g}}
      f_1(y) (f_2 \cdot f_3)(y')  \\
  &=  \frac{1}{\card{B}^2}
      \sum_{\substack{y, y' \in G \\ y y' = g}}
      f_1(y)
      \sum_{\substack{y'', y''' \in G \\ y'' y''' = y'}}
      f_2(y'') f_3(y''')  \\
  &=  \frac{1}{\card{B}^2}
      \sum_{\substack{y, y', y'' \in G \\ y y' y'' = g}}
      f_1(y) f_2(y') f_3(y'') \,.
\end{align*}
We also find in the same way that
\[
    ((f_1 \cdot f_2) \cdot f_3)(g)
  = \frac{1}{\card{B}^2}
    \sum_{\substack{y, y', y'' \in G \\ y y' y'' = g}}
    f_1(y) f_2(y') f_3(y'') \,,
\]
which then shows that~$f_1 \cdot (f_2 \cdot f_3) = (f_1 \cdot f_2) \cdot f_3$.

We now show that
\begin{align*}
  I_B \cdot I_B       &=  I_B \,, \\
  I_B \cdot I_{B'}    &=  I_{B'} \,,  \\
  I_{B'} \cdot I_B    &=  I_{B'} \,,  \\
  I_{B'} \cdot I_{B'} &=  q I_B + (q-1) I_{B'} \,.
\end{align*}
With this we have achieved the following things:
\begin{itemize}
  \item
    This describes the multiplication of~$H_q$ in terms of the given basis of~$H_q$.
  \item
    It shows that~$I_B$ is the multiplicative neutral element for~$H_q$.
  \item
    We see that for~$q = 1$ we get the group algebra~$\Complex[\Integer/2]$.
\end{itemize}

Indeed, it holds for every~$f \in H_q$ that
\begin{align*}
      (I_B \cdot f)(g)
  &=  \frac{1}{\card{B}}
      \sum_{y \in G}
      I_B(y) f(y^{-1} g)  \\
  &=  \frac{1}{\card{B}}
      \sum_{b \in B}
      f(b^{-1} g)
   =  \frac{1}{\card{B}}
      \sum_{b \in B}
      f(g)
   =  \frac{\card{B}}{\card{B}} f(g)
   =  f(g)
\end{align*}
and similarly that
\begin{align*}
      (f \cdot I_B)(g)
  &=  \frac{1}{\card{B}}
      \sum_{y \in G}
      f(y) I_B(y^{-1} g)
   =  \frac{1}{\card{B}}
      \sum_{y \in G}
      f(gy) I_B((gy)^{-1} g)  \\
  &=  \frac{1}{\card{B}}
      \sum_{y \in G}
      f(gy) I_B(y^{-1})
   =  \frac{1}{\card{B}}
      \sum_{y \in G}
      f(gy^{-1}) I_B(y) \\
  &=  \frac{1}{\card{B}}
      \sum_{b \in B}
      f(gb)
   =  \frac{1}{\card{B}}
      \sum_{b \in B}
      f(g)
   =  \frac{\card{B}}{\card{B}} f(g)
   =  f(g) \,.
\end{align*}
We have for every~$b \in B$ that
\begin{align}
      (I_{B'} \cdot I_{B'})(b)
  &=  \frac{1}{\card{B}}
      \sum_{y \in G}
      I_{B'}(y) I_{B'}(y^{-1} b)  \notag  \\
  &=  \frac{1}{\card{B}}
      \sum_{b' \in B'}
      I_{B'}(b'^{-1} b) \notag  \\
  &=  \frac{1}{\card{B}}
      \sum_{b' \in B'}
      \underbrace{I_{B'}(b'^{-1})}_{=1} \notag  \\
  &=  \frac{\card{B'}}{\card{B}}  \label{is constant} \\
  &=  q \label{counting quotient} \,,
\end{align}
where we use for~\eqref{is constant} that~$b' \in B$ if and only if~$b'^{-1} \in B$, and therefore
\[
        b'^{-1} \in B'
  \iff  b'^{-1} \notin B
  \iff  b' \notin B
  \iff  b' \in B' \,,
\]
and we use for~\eqref{counting quotient} that
\[
    \frac{\card{B'}}{\card{B}}
  = \frac{\card{G} - \card{B}}{\card{B}}
  = \frac{\card{G}}{\card{B}} - 1
  = (q + 1) - 1
  = q \,.
\]
Lastly, we have for~$b' \in B'$ that
\begin{align}
      (I_{B'} \cdot I_{B'})(b')
  &=  \frac{1}{\card{B}}
      \sum_{y \in G}
      I_{B'}(y) I_{B'}(y^{-1} b') \notag  \\
  &=  \frac{1}{\card{B}}
      \sum_{b'' \in B'}
      I_{B'}(b''^{-1} b') \notag  \\
  &=  \frac{1}{\card{B}}
      \sum_{\substack{b'' \in B' \\ b'' \notin b' B}}
      \underbrace{I_{B'}(b''^{-1} b')}_{=1} \label{change sum}  \\
  &=  \frac{\card{B' \smallsetminus b' B}}{\card{B}}  \notag  \\
  &=  \frac{\card{B'} - \card{b' B}}{\card{B}} \label{difference}  \\
  &=  \frac{\card{B'} - \card{B}}{\card{B}} \notag  \\
  &=  \frac{\card{B'}}{\card{B}} - 1  \notag  \\
  &=  q - 1 \notag  \,,
\end{align}
where use for~\eqref{change sum} that
\[
        b''^{-1} b' \notin  B
  \iff  b'' B \neq b' B
  \iff  b'' \notin b' B \,,
\]
and use for~\eqref{difference} that~$b' B \subseteq B'$ because~$b' \notin B$.


\begin{remark}
  One can show more generally that
  \[
      B \backslash G / B
    = \coprod_{\sigma \in S_n} B P_\sigma B \,,
  \]
  where~$P_\sigma$ is the permutation matrix associated to~$\sigma \in S_n$.
  This decomposition is known as the \emph{Bruhat decomposition} of~$\GL_n(k)$, and can be seen as a version of the Gauß algorithm.
  
  This then shows that~$H_q$ has a basis indexed by the symmetric group~$S_n$.
  The algebra~$H_q$ is known as the \emph{Hecke algebra} of~$S_n$, and can be seen as the deformation of the groups algebra~$\Complex[S_n]$ along a parameter~$q$.
\end{remark}

































