\section{}





\subsection{}

It follows from the Artin--Wedderburn theorem that
\[
  \Complex[Q]
  \cong
  \mat{n_1}{\Complex} \times \dotsb \times \mat{n_r}{\Complex}
\]
where~$n_1, \dotsc, n_r$ are the dimensions of the irreducible complex representations of~$Q$.
We have for the resulting decomposition
\[
  8
  =
  \dim \Complex[Q]
  =
  n_1^2 + \dotsb + n_r^2
\]
three possibilities:
\[
  8
  =
  4 + 4 \,,
  \quad
  8
  =
  4 + 1 + 1 + 1 + 1 \,,
  \quad
  8
  =
  1 + \dotsb + 1  \,.
\]
The group~$Q$ acts trivially on~$\Complex$, and this makes~$\Complex$ into an irreducible complex representation.
This shows that~$n_i = 1$ for some~$i$, thus we can exclude the possiblity~$8 = 4 + 4$.
In the case~$8 = 1 + \dotsb + 1$ we would find that the algebra
\[
  \Complex[Q]
  \cong
  \Complex
  \times
  \dotsb
  \times
  \Complex
\]
is commmutative, and hence that the group~$Q$ abelian.
But this is not the case.
We are only left with the possiblity~$8 = 4 + 1 + 1 + 1 + 1$, which means that
\[
  \Complex[Q]
  \cong
  \Complex
  \times
  \Complex
  \times
  \Complex
  \times
  \Complex
  \times
  \mat{2}{\Complex} \,.
\]


\begin{remark}
  We have only used that the group~$Q$ is \dash{non}{commutative} and of order~$8$.
  If~$D$ is the dihedral group of order~$8$ then we therefore find that also
  \[
    \Complex[D]
    \cong
    \Complex
    \times
    \Complex
    \times
    \Complex
    \times
    \Complex
    \times
    \mat{2}{\Complex} \,.
  \]
  We hence find that~$\Complex[Q] \cong \Complex[D]$ even though~$Q \ncong D$.%
  \footnote{This can be fixed by endowing the group algebra with the additional structure of an Hopf algebra.}
\end{remark}





\subsection{}


There exists a unique linear map~$f \colon \Real[Q] \to \Quaternion$ with~$f(q) = a_q$ for every~$q \in Q$.
This linear map is multiplicative on the basis~$Q$ of~$\Real[Q]$ and is therefore an algebra homomorphism.
The basis~$1$,~$i$,~$j$,~$k$ of~$\Quaternion$ is contained in the image of~$f$, which shows that~$f$ is surjective.





\subsection{}

The center of~$Q$ is given by~$\gcenter(Q) = \{1, -1\}$.
Any two elements of~$Q$ commute up to sign, so the quotient group~$Q/\gcenter(Q)$ is abelian.
The group~$Q/\gcenter(Q)$ has order~$4$ and every nontrivial element has order~$2$.
It follows that
\[
  Q/\gcenter(Q)
  \cong
  \Integer/2 \times \Integer/2  \,.
\]
One such isomorphism is given by
\[
  \pm 1 \mapsto (0,0) \,, \quad
  \pm i \mapsto (0,1) \,, \quad
  \pm j \mapsto (1,0) \,, \quad
  \pm k \mapsto (1,1) \,.
\]
There exist four group homomorphisms~$\Integer/2 \times \Integer/2 \to \Real^2$, given by
\[
  \begin{cases}
    (1,0) \mapsto 1 \,, \\
    (0,1) \mapsto 1 \,,
  \end{cases}
  \quad
  \begin{cases}
    (1,0) \mapsto           -1  \,, \\
    (0,1) \mapsto \phantom{-}1  \,,
  \end{cases}
  \quad
  \begin{cases}
    (1,0) \mapsto \phantom{-}1  \,, \\
    (0,1) \mapsto           -1  \,,
  \end{cases}
  \quad
  \begin{cases}
    (1,0) \mapsto -1  \,, \\
    (0,1) \mapsto -1  \,.
  \end{cases}
  \quad
\]
We hence get four different group homomorphisms~$Q \to \Real^\times$.
These homomorphisms give four nonisomorphic \dash{one}{dimensional} real representations of~$Q$.





\subsection{}


It follows from the Artin--Wedderburn theorem that
\begin{equation}
  \label{RQ decomposition}
  \Real[Q]
  \cong
  \mat{n_1}{D_1} \times \dotsb \times \mat{n_r}{D_r}
\end{equation}
where~$D_1^\op, \dotsc, D_n^\op$ are the the endomorphism ring of the irreducible real representations of~$Q$.
We have previously found four nonisomorphic \dash{one}{dimensional} real representations of~$Q$;
their endomorphism rings are given by~$\Real$ (because these representations are \dash{one}{dimensional}).

The quaternions~$\Quaternion$ become an~\module{$\Real[Q]$}, i.e.\ a representation of~$Q$, via the constructed algebra homomorphism~$\Real[Q] \to \Quaternion$.
The quaternions~$\Quaternion$ are simple as an~\module{$\Quaternion$} because~$\Quaternion$ is a skew field, and hence simple as an~\module{$\Real[Q]$} because the homomorphism~$\Real[Q] \to \Quaternion$ is surjective.
Thus~$\Quaternion$ is another irreducible real representation of~$Q$;
it is nonisomorphic to the previous four irreducible representations for dimension reasons.
We find with the surjectivity of the homomorphism~$\Real[Q] \to \Quaternion$ that
\[
  \End_Q(\Quaternion)
  =
  \End_{\Real[Q]}(\Quaternion)
  =
  \End_\Quaternion(\Quaternion)
  \cong
  \Quaternion^\op
  \cong
  \Quaternion \,,
\]
where the last isomorphism is given by~$x \mapsto \conjugate{x}$.

We have found for the decomposition~\eqref{RQ decomposition} that, up to reordering,~$D_1 = \dotsb = D_4 = \Real$ and~$D_5 = \Quaternion$.
For dimension reasons there can’t be any more factors and~$n_i = 1$ for all~$i$.
Therefore
\[
  \Real[Q]
  \cong
  \Real \times \Real \times \Real \times \Real \times \Quaternion \,.
\]


\begin{remark}
  We have for the dihedral group~$D$ of order~$8$ that
  \[
    \Real[D]
    \cong
    \Real \times \Real \times \Real \times \Real \times \mat{2}{\Real}  \,.
  \]
  Hence~$\Real[D] \ncong \Real[Q]$ even though~$\Complex[D] \cong \Complex[Q]$.
  
  Indeed, the abelianization of~$D$ is given by~$\Integer/2 \times \Integer/2$.
  We therefore find as for the quaternion group~$Q$ that~$D$ admits precisely four nonisomorphic real \dash{one}{dimensional} representations.
  Their endomorphism rings are given by~$\Real$.
  
  The dihedral group also acts on~$\Real^2 \cong \Complex$ in the usual ways.
  The endomorphisms of this action are those~\dash{$\Real$}{linear} maps~$f \colon \Complex \to \Complex$ that are compatible with rotation by~$90^\circ$, i.e.\ multiplication with~$i$, and reflection at the real axis, i.e.\ complex conjugation.
  The first conditions ensures that~$f$ is already~\dash{$\Complex$}{linear}, and hence given by multiplication with some complex number~$z \in \Complex$.
  The second conditions tells us that already~$z \in \Real$.
  This shows that
  \[
    \End_{D}(\Complex)
    =
    \Real \,.
  \]
  
  For the Artin--Wedderburn decomposition
  \[
    \Real[D]
    \cong
    \mat{n_1}{D_1} \times \dotsb \times \mat{n_r}{D_r}
  \]
  we now find that, up to reordering,~$D_1 = \dotsb = D_5 = \Real$.
  We also know  that
  \[
    n_i \dim D_i
    =
    \dim V_i
  \]
  because~$D_i^{n_i}$ is the (up to isomorphism unique) simple module of~$\mat{n_i}{D_i}$.
  We hence find that~$n_1 = n_2 = n_3 = n_4 = 1$ and~$n_5 = 2$.
  We find by dimension reasons that there can be no other factors, and hence that
  \[
    \Real[D]
    \cong
    \Real \times \Real \times \Real \times \Real \times \mat{2}{\Real}  \,.
  \]
\end{remark}






















